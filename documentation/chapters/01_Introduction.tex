

\chapter{Bevezetés}
\label{chap:01_Intro}

\section{Téma leírása}
\label{sec:theme_def}

A szoftverfejlesztésben általában egy fő platformra specializálódunk, és ebben fejlesztünk. Ez praktikus döntés, hiszen a terület olyan gyorsan fejlődik, hogy nem lehet minden nyelven és platformon mély és naprakész tudást fenntartani. Emiatt azonban a fejlesztők hajlamosak nagyon egyoldalúan gondolkodni, csak a saját maguk által preferált környezet kínálta megoldásokat, technikákat használni és elfogadni. 

A téma lehetőséget kínál az érdeklődő hallgatóknak, hogy több, jelentősen különböző nyelvekbe is belekóstoljanak, és ezáltal jobban megismerjék a manapság elérhető programozási nyelvek kínálta technikákat és lehetőségeket. A választott nyelveken olyan programokat kell elkészíteni, amelyek elég egyszerűek ahhoz, hogy legyen idő több programozási nyelven is megírni őket, ugyanakkor elég bonyolultak is legyenek, hogy a választott nyelvek eszköztárát alkalmazni lehessen rajtuk. Ilyen lehet például a black jack kártyajáték vagy a game of life. 

A munka kimenete, hogy a hallgató bár nem feltétlen lesz "profi" egyik nyelvben sem, egy magas szintű rálátással rendelkezik majd az egyes nyelvekre és platformokra. Ez szolgálhat alapul a későbbi választáshoz, illetve akár konklúziók is megfogalmazhatók, hogy melyik nyelv milyen problémákban erős. 

Javsolt nyelvek: Java, Kotlin, Python, JavaScript, Scala, Haskell, Erlang, Lua. A feladatnak az is része, hogy a használható fejlesztőkörnyezeteket és library-kat is feltérképezzük, ezért a feladatokat TDD megközelítéssel kell elvégezni.

A téma továbbvihető későbbi félévekre, sőt, igényes kidolgozása több félévet igényel.

\section{A projekt célja}
\label{sec:goal}

Alapvetően azért választottam ezt a témát, mert szerettem volna kicsit elrugaszkodni az eddig használt nyelvek világától és megismerni más, akár típusban eltérő, nyelveket is. A másik oka, hogy ezt a témát választottam, pedig az, hogy több nyelv megismerésével talán konkretizálódik, hogy mivel szeretnék foglalkozni a jövőben. Úgy gondolom, ha egy programozási nyelv megtetszik, akkor sokkal könnyebb témát keresni hozzá, hiszen ismertek a lehetőségei.

A célom pedig az volt, hogy tágítsam a látókörömet és ezáltal talán kialakul majd egy kép bennem, hogy milyen irányban induljak el a jövőben. A témalaborig csupán csak azokkal a nyelvekkel ismerkedtem és foglalkoztam, amiket az egyetemi évek alatt oktattak. Fontos volt számomra, hogy tágítsam a látókörömet. Persze már a választásomkor tudtam, hogy egy félév nem lesz elegendő, hogy megismerjek minden nyelvet, amit szeretnék. De úgy vélem, hogy kiválasztva a két legszimpatikusabbat, azért jelentős tapasztalatokkal gazdagodhatok, akkor is, ha tudom, hogy még rengeteg más nyelv van, amit még érdemes kipróbálni.

Ahogy a témaleírás\ref{sec:theme_def}~is írja, ez a téma tipikusan továbbvihető a későbbi félévekre, hiszen nagy rész teljesen önálló munka. Én úgy érzem a lényegét mégis átadta, hiszen bár sok különböző nyelv van, azért az egyes típusokban sok a hasonlóság (gondolok itt például az OO vagy a szkript nyelvek világára, akár a funkcionális programozásra). Alapvetően ez egy remek téma és lehetőség volt, hogy betekintést nyerjek más típusú nyelvek világába és megismerhessem sajátosságaikat. 
  
\section{Választott nyelvek}
\label{sec:choosed_lang}

Már a témalabor választásakor elgondolkoztam, hogy melyik nyelvek legyen azok, amelyekkel egy féléven át foglalkozni szeretnék. Több elképzelésem is volt, de nem akartam túlzásba esni. Ezért két szempont szerint elkezdtem szortírozni a lehetőségeket. 

Először is, mindenképpen szerettem volna választani egy OO nyelvet, amit még sosem használtam. A választásom pedig azért a Kotlin\ref{chap:03_kotlin}~lett. Ennek több oka is volt. A barátaim közül többen is javasolták, hogy válasszam ezt, mert ők egy nyári projekt alkalmával kipróbálták és rendkívül tetszett nekik. Ösztönöztek, hogy próbáljam ki én is. A másik ok, pedig egy kis utánajárás következtében alakult ki. Pár bemutatóvideó után nekem is megtetszett ez a nyelv, mert bár Java alapokra van helyezve, mégis szinte minden hibáját kijavították benne. Sőt, ha kicsit pongyolán akarok fogalmazni, akkor összegyúrták a Java és a C\verb|#| nyelvet és az ő gyermekük lenne a Kotlin. Tehát ezzel mindenképp meggyőzött, hogy ez legyen az egyik választott nyelvem.

A másik szempontom pedig, hogy válasszak egy nem OO nyelvet. Először még nem tudtam eldönteni, hogy szkript vagy deklaratív nyelvet válasszak, így csak egy sorrendet tudtam kialakítani, hogy előbb lenne egy szkript nyelv és aztán egy deklaratív. 

A szkript nyelvek közül nehéz volt választani. Bár szerencsére sok a hasonlóság, azért mindegyiknek megvannak a maga sajátosságai. Ilyenkor persze mindenkinek először a JavaScript jut eszébe és én épp ezért gondolkodtam valami másban. JavaScript betekintőt a félév során egy másik tantárgyból úgyis kapunk, ezért is gondoltam úgy, hogy eltérek a ''mainstream'' választástól. Épp emiatt a második nyelvemnek a Lua-t\ref{chap:02_lua}~választottam. 

Volt egy másik oka is, amiért ez lett a választás. Gyakran játszok ugyanis egy MMORPG játékkal, amelyben használhatok külső, kisegítő eszközöket (ún. Addonokat), és mindig is érdekelt, hogy ezeket hogyan csinálják, mert én is szeretnék csinálni számomra testre szabottakat vagy akár a meglévőket módosítani. Amikor utánajártam, hogy ezeket az Addonokat Lua-ban írják, akkor gondoltam úgy, hogy szeretnék foglalkozni ezzel a nyelvvel, ha lesz rá időm. És ezért esett a választás erre a nyelvre.

Mindazok a nyelvek, amiket nem választottam ki, de a munkám során vagy korábbi évek alatt előkerültek (Java, C, C\verb|++|, C\verb|#|, JavaScript, Pascal, Python), szeretném majd (ha tudom) referenciaként vagy akár ''harmadik szemszögként'' felhasználni. Így a nyelvek bemutatását még több perspektívával tudom ellátni és bemutatni, hogy mennyi apróbb különbség/hasonlóság van a nyelvekben.

Deklaratív nyelvet azért is raktam a végére, mivel a 7.~félévben lehetőség lesz egy elágazó tantárgy kereteiben tanulni róla. Mindemellett úgy gondoltam, ha jut rá időm, szeretném kipróbálni a Haskell-t. Sajnos azonban ez a terv nem tudott megvalósulni a félév alatt.


\section{A projektek bemumtatása}

Először bemutatom a választott nyelveket vázlatosan. Az egyes példánál a projektekből idézek, illetve megpróbálok minél több referenciát felhozni, hogy más nyelvekben milyen hasonlóságok és különbségek vannak.. Majd ezután egy összehasonlítást mutatok be mind a BlackJack, mind a GameOfLife projektekkel kapcsolatban, melyben a két nyelv különbségeit és hasonlóságait mérem össze.
 
